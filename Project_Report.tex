% Generated by GrindEQ Word-to-LaTeX 
\documentclass{article} % use \documentstyle for old LaTeX compilers

\usepackage[utf8]{inputenc} % 'cp1252'-Western, 'cp1251'-Cyrillic, etc.
\usepackage[english]{babel} % 'french', 'german', 'spanish', 'danish', etc.
\usepackage{amsmath}
\usepackage{amssymb}
\usepackage{txfonts}
\usepackage{mathdots}
\usepackage[classicReIm]{kpfonts}
\usepackage{graphicx}

% You can include more LaTeX packages here 


\begin{document}

%\selectlanguage{english} % remove comment delimiter ('%') and select language if required


\noindent \textbf{    \underbar{Project for professor course allotment optimization}}

\noindent \textbf{}

\noindent \textbf{                                                                                 \underbar{BY}}

\noindent \textbf{\underbar{}}

\noindent \textbf{\underbar{Name(s) of the Student(s) }                                                                                                    \underbar{ID.No.(s)}}

\noindent Nishanth M S                                                                                                               2021B3A72176G

\noindent Anshul Goel                                                                                                                  2021B3A72863G

\noindent Satvik Narula                                                                                                                2021B3A72482G

\noindent 

\noindent                                                                      

\noindent 

\noindent 

\noindent 

\noindent                                \includegraphics*[width=3.81in, height=1.52in]{image1}

\textbf{\underbar{}}

\textbf{\underbar{}}

\textbf{\underbar{}}

\textbf{\underbar{}}

\noindent \textbf{\underbar{}}

\noindent \textbf{\underbar{}}

\noindent \textbf{\underbar{}}

\noindent \textbf{\underbar{Introduction}}

\noindent The Courses allotment Planner project aims to efficiently allocate courses to instructors based on their preferences and the constraints provided. This report provides an overview of the project, its objectives, methodology, and outcomes.

\noindent 

\noindent \textbf{\underbar{Project Overview}}

\noindent 

\noindent \textbf{Objective Function}

\noindent \textbf{1. Allocate Courses:} The primary goal is to allocate the specified courses to instructors in three distinct groups: X1, X2, and X3.

\noindent \textbf{2. Preference Collection:} Allow instructors to provide their preference order for courses to enhance satisfaction and engagement.

\noindent \textbf{3. Prioritize Compulsory Courses:} Prioritize the allocation of CDCs, HDCDCs

\noindent 

\noindent \textbf{Inputs}

\noindent \textbf{1. Total Courses: }

\noindent     - Total CDCs: 22 (can be different)

\noindent     - Total Electives: 12 (can be different)

\noindent     - Total HDCDCs: 12 (can be different)

\noindent     - Total HDEs: 12 (can be different)

\noindent 

\noindent \textbf{2. Instructor Information:}

\noindent     - Total X1 Instructors: no specific number as such, but expected to be a less number

\noindent     - Total X2 Instructors: no specific number

\noindent     - Total X3 Instructors: no specific number

\noindent 

\noindent 

\noindent 

\noindent 

\noindent 

\noindent  \textbf{Output}

\noindent The output is a finalized table that ensures equitable distribution of courses among instructors, considering their preferences and the compulsory course requirements.

\noindent 

\noindent \textbf{\underbar{Methodology}}

\noindent \textbf{1. Input Collection}

\noindent \textbf{We input a text file that contains the list of courses on the first line along with their course code. According to the type of the course, they have been labelled cdc,HDcdc , HDel, el.}

\noindent \textbf{}

\noindent \textbf{2. Preference Collection}

\noindent \textbf{The subsequent lines of the text file are in the format of [professor\_name]   [professor capacity(0.5/1/1.5) ]  and preference list in order }

\noindent \textbf{}

\noindent \textbf{3. Allocation Logic}

\noindent We have used the Ford -Fulkerson (Edmond-Karp) algorithm to allocate the courses to the professors. There are 4 set of nodes, Source, Courses, Professors, and Sink.  We have added a capacity of 2 to all edges from source to all the courses and then we have added edges from courses to professors according to the preference of the professors with a capacity of 2.  We have added edges from professor to Sink with capacities relative to their course allocation. If a professor is x1(teaches 0.5) , they have been given capacity of 1. Similarly, professor in x2(teaches 1) have been given capacity of 2 and professors in x3(teaches 1.5) have been given capacity of 3. The program then allots courses to professors based on the graph formed (shown below)

\noindent \textbf{                   }\includegraphics*[width=4.92in, height=3.38in]{image2}\textbf{}

\noindent \textbf{}

\noindent \textbf{\underbar{Conclusion}}

\noindent In this project, we learned about graphs and their versatility in solving problems of different types. while many different algorithms like Shortest path, bipartite, Ford-Fulkerson could be used, we have used Ford Fulkerson for its simplicity and ease. The code successfully allots courses to the professors taking into account their varied teaching capacities and preferences as much as possible.

\noindent \textbf{}

\noindent \textbf{\underbar{References:}}

\noindent \textbf{---https://graphonline.ru/en/}

\noindent https://networkx.org/documentation/stable/reference/index.html\textbf{}

\noindent https://docs.python.org/3/tutorial/inputoutput.html\textbf{}

\noindent https://algorithms.discrete.ma.tum.de/graph-algorithms/flow-ford-fulkerson/index\_en.html\textbf{}

\noindent 


\end{document}

